% !TEX TS-program = lualatex
% !TEX encoding = UTF-8

\documentclass[psautier_nocturne_fr.tex]{subfiles}

\ifcsname preamble@file\endcsname
  \setcounter{page}{\getpagerefnumber{M-pnfr10_ordinaire_psautier}}
\fi

\begin{document}
\feast{OR}{Ordinaire de l'Office Divin\\à Matines}{Ordinaire}{Ordinaire}{1}{}{}{}{}{}{}
\addcontentsline{toc}{chapter}{Ordinaire de l'Office Divin à Matines}

\intermediatetitle{Avant l'Office}

\begin{paracol}{2}
\lettrine{A}{peri}, Dómine, os meum ad benedicéndum nomen sanctum tuum: munda quoque cor meum ab ómnibus vanis, pervérsis et aliénis 
cogitatiónibus; intelléctum illúmina, afféctum inflámma, ut digne, atténte ac devóte hoc officium recitáre váleam, et exaudíri mérear
ante conspéctum divinae Majestátis tuae. Per Christum Dóminum nostrum. Amen.

\switchcolumn

\lettrine{O}{uvre} ma bouche, Seigneur, afin qu’elle bénisse ton saint nom, purifie aussi mon cœur de toute pensée
vaine, mauvaise, étrangère. Éclaire mon intelligence, enflamme mon amour,
afin que je puisse réciter cet office avec respect, attention et dévotion, et mériter d’être exaucé en
présence de ta divine majesté. Par le Christ, notre Seigneur. Amen.

\switchcolumn*

\lettrine{D}{ómine}, in unióne illíus divínæ intentiónis, qua ipse in terris laudes Deo persolvísti, has tibi horas \rubric{(vel} hanc tibi horam\rubric{)} persólvo.

\switchcolumn

\lettrine{S}{eigneur}, en union avec ces divines intentions que tu avais toi-même sur terre lorsque
tu louais Dieu, je t’offre cette \rubric{(}ces\rubric{)} heure\rubric{(}s\rubric{)}.

\switchcolumn*

\lettrine{P}{ater noster}, qui es in cælis, sanctificétur nomen tuum. Advéniat regnum tuum. Fiat volúntas tua, sicut in cælo et in terra.
Panem nostrum quotidiánum da nobis hódie. Et dimítte nobis débita nostra, sicut et nos dimíttimus debitóribus nostris. Et ne nos
indúcas in tentatiónem: sed líbera nos a malo. Amen.

\switchcolumn

\lettrine{N}{otre Père}, qui es aux cieux, que ton nom soit sanctifié, que ton règne vienne,
que ta volonté soit faite sur la terre comme au ciel.
Donne-nous aujourd’hui notre pain de ce jour. Pardonne-nous nos offenses, comme nous pardonnons aussi à ceux qui nous ont offensés.
Et ne nous laisse pas entrer en tentation mais délivre-nous du Mal. Amen.

\switchcolumn*

\lettrine{A}{ve Maria}, grátia plena, Dóminus tecum: benedíctus fructus ventris tui Jesus. Sancta María, Mater Dei, ora pro nobis 
peccatóribus, nunc et in hora mortis nostræ. Amen.

\switchcolumn

\lettrine{J}{e vous salue Marie}, pleine de grâce, le Seigneur est avec vous.
Vous êtes bénie entre toutes les femmes et Jésus, le fruit de vos entrailles, est béni.
Sainte Marie, Mère de Dieu, priez pour nous pauvres pécheurs, maintenant et à l’heure de notre mort.
Amen.

\switchcolumn*

\lettrine{C}{redo in Deum}, Patrem omnipoténtem, Creatórem cæli et terræ. Et in Jesum Christum, Fílium ejus únicum, Dóminum nostrum,
qui concéptus est de spíritu Sancto, natus ex María Virgine, passus sub Póntio Piláto, crucifixus, mórtuus et sepúltus: descéndit
ad ínferos: tértia die resurréxit a mórtuis; ascéndit ad cælos, sedet ad déxteram Patris omnipoténtis: inde ventúrus est judicáre
vivos et mórtuos. Credo in Spíritum sanctum, sanctam Ecclésiam cathólicam, Sanctórum communiónem, remissiónem peccatórum,
carnis resurrectiónem, vitam ætérnam. Amen.

\switchcolumn

\lettrine{J}{e crois en Dieu}, le Père tout-puissant, créateur du ciel et de la terre.
Et en Jésus Christ, son Fils unique, notre Seigneur;
qui a été conçu du Saint Esprit, est né de la Vierge Marie,
a souffert sous Ponce Pilate, a été crucifié,
est mort et a été enseveli, est descendu aux enfers;
le troisième jour est ressuscité des morts,
est monté aux cieux, est assis à la droite de Dieu le Père tout-puissant,
d’où il viendra juger les vivants et les morts.
Je crois en l’Esprit Saint, à la sainte Église catholique, à la communion des saints,
à la rémission des péchés, à la résurrection de la chair, à la vie éternelle. Amen.

\end{paracol}

\intermediatetitle{Ouverture de l'Office}

\gscore{ORIa}{T}{}{Domine labia mea!Tonus simplex}{Seigneur, ouvre mes lèvres, et ma bouche annoncera ta louange.\\\\
Dieu, viens à mon aide, Seigneur, hâte-toi de me secourir.\\\\
Gloire au Père, et au Fils, et au Saint-Esprit\\
comme il était au commencement, maintenant et toujours, pour les siècles des siècles. Amen.\\
Alléluia.\\
\rubric{Septuagésime et Carême:} Louange à toi, Seigneur, Roi d'éternelle gloire.}
\gscore{ORIb}{T}{}{Domine labia mea!Tonus festivus}{\rubric{Ton solennel}\\
Dieu, viens à mon aide, Seigneur, hâte-toi de me secourir.\\
Gloire au Père, et au Fils, et au Saint-Esprit\\
comme il était au commencement, maintenant et toujours, pour les siècles des siècles. Amen.\\
Alléluia.\\
\rubric{Septuagésime et Carême:} Louange à toi, Seigneur, Roi d'éternelle gloire.}

\smalltitle{Invitatoire}

\rubric{Antienne au Psautier, au Propre, ou au Commun.}

\label{M-ORIP2}\label{M-ORIP3a}\label{M-ORIP3b}\label{M-ORIP4a}\label{M-ORIP4b}\label{M-ORIP4c}\label{M-ORIP5}\label{M-ORIP6a}\label{M-ORIP6b}\label{M-ORIP6c}\label{M-ORIP7a}\label{M-ORIP7b}
\psaume{94}{VLrepet}

\smalltitle{Hymne}

\rubric{Hymne au Psautier, au Propre, ou au Commun.}

\nocturn{1}

\smalltitle{Psalmodie}

\rubric{Antiennes, Psaumes et Versicule au Psautier, au Propre, ou au Commun.}

\gscore[n]{ORPN}{T}{}{Pater Noster}{Notre Père...\\Et ne nous laisse pas entrer en tentation.\\Mais délivre-nous du Mal.}

\smalltitle{Absolution}

\gscore[n]{ORA}{T}{}{Absolutio}{Seigneur Jésus-Christ, exauce les prières de tes serviteurs, et aie pitié de nous,
toi qui vis et règnes avec le Père et le Saint-Esprit, dans les siècles des siècles.\\ \rubric{\rrrub} Amen.}

\smalltitle{Bénédictions et Leçons}

\gscore[n]{ORLb}{T}{}{Benedictio!Tonus simplex}{Veuillez, Maître, bénir.\\
\rubric{Bén.} Que le Père éternel nous bénisse d'une bénédiction perpétuelle.\\
\rubric{\rrrub} Amen.}

\gscore[n]{ORLc}{T}{}{Benedictio!Tonus solemnis}{\rubric{Ton solennel}\\Veuillez, Maître, bénir.\\
\rubric{Bén.} Que le Père éternel nous bénisse d'une bénédiction perpétuelle.\\
\rubric{\rrrub} Amen.}

\rubric{Quand l'officiant n'est pas au moins diacre, ou qu'on est seul, on dit \normaltext{Jube, Dómine, benedícere}
et on ajoute la bénédiction.}
\pagebreak

\rubric{À la fin des lectures, le lecteur ajoute:}
\gscore[n]{ORLd}{T}{}{In fine lectionum!Tonus simplex}{Et toi, Seigneur, aie pitié de nous. \rubric{\rrrub} Nous rendons grâces à Dieu.}
\gscore[n]{ORLe}{T}{}{In fine lectionum!Tonus solemnis}{\rubric{Ton solennel}\\Et toi, Seigneur, aie pitié de nous. \rubric{\rrrub} Nous rendons grâces à Dieu.}

\rubric{Bénédictions pour les autres lectures:}

\begin{paracol}{2}

\rubric{\emph{Benedictio 2.}} Unigénitus \textit{Dei} \textbf{Fí}lius~\GreSpecial{*}
nos benedícere et adjuváre dignétur.
\hspace{\specialcharhsep}\rr Amen.

\switchcolumn

\rubric{\emph{Bén. 2.}} Que le Fils unique de Dieu daigne nous bénir et nous secourir.
\hspace{\specialcharhsep}\rr Amen.

\switchcolumn*

\rubric{\emph{Benedictio 3.}} Spíritus \textit{Sancti} \textbf{grá}tia~\GreSpecial{*}
illúminet sensus et corda nostra.
\hspace{\specialcharhsep}\rr Amen.

\switchcolumn

\rubric{\emph{Bén. 3.}} Que la grâce du Saint-Esprit illumine nos esprits et nos cœurs.
\hspace{\specialcharhsep}\rr Amen.

\end{paracol}

\nocturn{2}

\rubric{Comme au premier nocturne, sauf l'absolution et les bénédictions.}

\smalltitle{Absolution}

\begin{paracol}{2}

\rubric{\emph{Absolutio 2.}}
Ipsíus píetas et misericódi\textit{a nos} \textbf{ád}juvet,~\GreSpecial{*}
qui cum Patre et Spíritu Sancto vivit et regnat in sǽcula sæculórum.
\hspace{\specialcharhsep}\rr Amen.

\switchcolumn

\rubric{\emph{Absolution 2.}}
Qu'il nous secoure par sa bonté et sa miséricorde, celui qui, avec le Père et le Saint-Esprit, vit et règne dans les siècles des siècles.
\hspace{\specialcharhsep}\rr Amen.

\end{paracol}

\smalltitle{Bénédictions et Leçons}

\begin{paracol}{2}

\rubric{\emph{Benedictio 4.}} Deus Pa\textit{ter om}\textbf{ní}potens~\GreSpecial{*}
sit nobis propítius et clemens.
\hspace{\specialcharhsep}\rr Amen.

\switchcolumn

\rubric{\emph{Bén. 4.}}
Que Dieu le Père tout-puissant soit pour nous propice et plein de clémence.
\hspace{\specialcharhsep}\rr Amen.

\switchcolumn*

\rubric{\emph{Benedictio 5.}} Chris\textit{tus per}\textbf{pé}tuæ~\GreSpecial{*}
det nobis gaúdia vitæ.
\hspace{\specialcharhsep}\rr Amen.

\switchcolumn

\rubric{\emph{Bén. 5.}}
Que le Christ nous donne les joies de l'éternelle vie.
\hspace{\specialcharhsep}\rr Amen.

\switchcolumn*

\rubric{\emph{Benedictio 6.}} Ignem su\textit{i a}\textbf{mó}ris~\GreSpecial{*}
accéndat Deus in córdibus nostris.
\hspace{\specialcharhsep}\rr Amen.

\switchcolumn

\rubric{\emph{Bén. 6.}}
Que Dieu daigne allumer dans nos cœurs le feu de son amour.
\hspace{\specialcharhsep}\rr Amen.

\end{paracol}

\nocturn{3}

\rubric{Comme au premier nocturne, sauf l'absolution et les bénédictions.}

\smalltitle{Absolution}

\begin{paracol}{2}

\rubric{\emph{Absolutio 3.}}
A vínculis peccató\textit{rum nos}\textbf{tró}rum~\GreSpecial{*}
absólvat nos omnípotens et miséricors Dóminus.
\hspace{\specialcharhsep}\rr Amen.

\switchcolumn

\rubric{\emph{Absolution 3.}}
Que le Dieu tout-puissant et miséricordieux daigne nous délivrer des liens de nos péchés.
\hspace{\specialcharhsep}\rr Amen.

\end{paracol}

\smalltitle{Bénédictions et Leçons}

\begin{paracol}{2}

\rubric{\emph{Benedictio 7.}}
Evangé\textit{lica} \textbf{léc}tio~\GreSpecial{*}
sit nobis salus et protéctio.
\hspace{\specialcharhsep}\rr Amen.

\switchcolumn

\rubric{\emph{Bén. 7.}}
Que la lecture du saint Évangile nous soit salut et protection.
\hspace{\specialcharhsep}\rr Amen.

\end{paracol}

\rubric{Le dimanche et aux fêtes du Seigneur:}

\begin{paracol}{2}

\rubric{\emph{Benedictio 8.}}
Diví\textit{num au}\textbf{xí}lium~\GreSpecial{*}
máneat semper nobíscum.
\hspace{\specialcharhsep}\rr Amen.

\switchcolumn

\rubric{\emph{Bén. 8.}}
Que le secours divin demeure toujours avec nous.
\hspace{\specialcharhsep}\rr Amen.

\end{paracol}

\pagebreak

\rubric{Aux fêtes de la Vierge Marie:}

\begin{paracol}{2}

\rubric{\emph{Benedictio 8.}}
Cujus \textit{festum} \textbf{có}limus,~\GreSpecial{*}
ipsa Virgo vírginum intercédat pro nobis ad Dóminum.
\hspace{\specialcharhsep}\rr Amen.

\switchcolumn

\rubric{\emph{Bén. 8.}}
Que celle dont nous célébrons la fête, la Vierge des vierges elle-même, intercède pour nous auprès du Seigneur.
\hspace{\specialcharhsep}\rr Amen.

\end{paracol}

\rubric{Aux fêtes des saints:}

\begin{paracol}{2}

\rubric{\emph{Benedictio 8.}}
Cujus \rubric{(vel} Quarum\rubric{)} \textit{festum} \textbf{có}limus,~\GreSpecial{*}
ipse \rubric{(vel} ipsa \rubric{aut} ipsæ\rubric{)}
intercédat \rubric{(vel} intercédant\rubric{)} pro nobis ad Dóminum.
\hspace{\specialcharhsep}\rr Amen.

\switchcolumn

\rubric{\emph{Bén. 8.}}
Que celui \rubric{(ou \normaltext{celle}, \normaltext{ceux}, \normaltext{celles})} dont nous célébrons la fête intercède\rubric{(}nt\rubric{)} pour nous auprès du Seigneur.
\hspace{\specialcharhsep}\rr Amen.

\end{paracol}

\begin{paracol}{2}

\rubric{\emph{Benedictio 9.}}
Ad societátem cívium \textit{super}\textbf{nó}rum~\GreSpecial{*}
perdúcat nos Rex Angelórum.
\hspace{\specialcharhsep}\rr Amen.

\switchcolumn

\rubric{\emph{Bén. 9.}}
Que le Roi des Anges nous fasse parvenir à la société des citoyens célestes.
\hspace{\specialcharhsep}\rr Amen.

\end{paracol}

\rubric{Si on lit un évangile à la dernière leçon en vertu d'une commémoraison:}

\begin{paracol}{2}

\rubric{\emph{Benedictio 9.}}
Per evangé\textit{lica} \textbf{dic}ta~\GreSpecial{*}
deleántur nostra delícta.
\hspace{\specialcharhsep}\rr Amen.

\switchcolumn

\rubric{\emph{Bén. 9.}}
Par les paroles de l'Évangile, que nos péchés soient effacés.
\hspace{\specialcharhsep}\rr Amen.

\end{paracol}

\intermediatetitle{Te Deum}

\rubric{Après la dernière Leçon, le dimanche et aux fêtes, sauf en la fête
des Saints Innocents, et aux dimanches des temps de l'Avent, de la Septuagésime, du Carême
et de la Passion, on chante l'hymne \normaltext{Te Deum}: ton solennel p.\ \pageref{ORTDa}, ton simple p.\ \pageref{ORTDb}.
Dans le cas contraire, on chante un dernier répons.
}

\intermediatetitle{Conclusion}

\rubric{Après le \normaltext{Te Deum} ou le dernier répons,
on commence les Laudes à partir du verset \normaltext{Deus, in adjutórium}.
Si les Matines ne sont pas immédiatement suivies des Laudes, l'officiant dit:}

\gscore[n]{ORDV}{T}{}{Dominus vobiscum}{Le Seigneur soit avec vous.\\\rubric{\rrrub} Et avec votre esprit.\\\rubric{\vvrub} Prions.}

\rubric{Ou bien, sur le même ton, si l'officiant n'est pas au moins diacre:}

\begin{paracol}{2}

\versiculus{Dómine, exáudi oratiónem meam.}{Et clamor meus ad te véniat.}
\vv Orémus.

\switchcolumn

\versiculus{Seigneur, exauce ma prière.}{Et que mon cri parvienne jusqu'à toi.}
\vv Prions.

\end{paracol}

\rubric{Puis il dit la collecte du jour et on répond \normaltext{Amen.}}

\begin{paracol}{2}

\versiculus{Dóminus vobíscum.}{Et cum spíritu tuo.}
\rubric{ou bien \normaltext{Dómine, exáudi}, etc.}\\
\rubric{\normaltext{Benedicámus Dómino}, ton des diférents jours p.\ \pageref{TCBD}.}
\versiculus{Fidélium ánimæ~\cc~per misericórdiam Dei requiéscant in pace.}{Amen.}

\switchcolumn

\versiculus{Le Seigneur soit avec vous.}{Et avec votre esprit.}
\rubric{ou bien \normaltext{Seigneur, exauce}, etc.}
\versiculus{Bénissons le Seigneur.}{Nous rendons grâces à Dieu.}
\versiculus{Que par la miséricorde de Dieu, les âmes des fidèles trépassés reposent en paix.}{Amen.}

\end{paracol}

\rubric{On finit par un \normaltext{Pater} entièrement en silence.}

\feast{F1}{Psautier nocturne du dimanche}{Psautier}{Psautier}{1}{}{}{}{}{}{}
\addcontentsline{toc}{chapter}{Psautier nocturne du dimanche}

\feast{AF1}{Avent}
	{Psautier}{Avent}{2}{}{}{}{}{}{}
\addcontentsline{toc}{section}{Avent}

\rubric{Les premier et deuxième dimanche.}
\gscore{A1F1I}{I}{}{Regem venturum Dominum}{Le Roi qui va venir, c'est le Seigneur, venez, adorons-le.}
\rubric{Les troisième et quatrième dimanche, sauf si la vigile de Noël tombe le dimanche.}
\gscore{A3F1I}{I}{}{Prope est jam Dominus}{Tout près est le Seigneur, venez, adorons-le.}

\gscore{A1Hmed}{H}{}{Verbum supernum prodiens}{%
\rubric{1.} Ô Verbe très-haut, tu parais,
lumière, tu jaillis du Père,
tu nais pour secourir le monde
quand le temps décline en sa course.\\\\
\rubric{2.} Éclaire maintenant les coeurs,
consume-les de ton amour,
qu'à l'annonce de ta venue
les péchés soient enfin bannis.\\\\
\rubric{3.} Et lorsque tu viendras en juge
sonder les actions et les cœurs,
peser ce qui était caché,
donner aux justes le Royaume.\\\\
\rubric{4.} Puissions-nous échapper aux peines
que notre crime a méritées:
fais-nous avec les bienheureux
partager ton ciel pour toujours.\\\\
\rubric{5.} Louange, honneur, force et gloire,
soient à Dieu le Père et au Fils
et de même au Paraclet,
à jamais dans tous les siècles.
Amen.}

\nocturn{1}
\gscore{A1F1N1A1}{A}{1}{Veniet ecce}{Voici que viendra le Roi, le Très-haut, avec une grande puissance, pour sauver les nations, alléluia.}
\psaume{1}{1}
\gscore[n]{A1F1N1A1}{A}{1}{}{}
\gscore{A1F1N1A2}{A}{2}{Confortate manus}{Fortifiez les mains languissantes, prenez courage et dites : Voici notre Dieu viendra et il nous sauvera, alléluia.}
\psaume{2}{2}
\gscore[n]{A1F1N1A2}{A}{2}{}{}
\gscore{A1F1N1A3}{A}{3}{Gaudete omnes}{Réjouissez-vous tous et livrez-vous à la joie, car voici que le Seigneur de la vengeance viendra, il amènera la rétribution, il viendra lui-même et nous sauvera.}
\psaume{3}{3}
\gscore[n]{A1F1N1A3}{A}{3}{}{}

\begin{paracol}{2}
\versiculus{Ex Sion spécies decóris ejus.}{Deus noster maniféste véniet.}
\switchcolumn
\versiculus{C’est de Sion que vient l’éclat de sa splendeur.}{Notre Dieu viendra et se manifestera.}
\end{paracol}

\nocturn{2}
\gscore{A1F1N2A1}{A}{4}{Gaude et laetare}{Réjouis-toi et livre-toi à la joie, fille de Jérusalem ; voici que ton Roi vient à toi ; Sion, ne crains pas, car ton salut viendra bientôt.}
\psaume{8}{4e}
\gscore[n]{A1F1N2A1}{A}{4}{}{}
\gscore{A1F1N2A2}{A}{5}{Rex noster adveniet}{Notre Roi, le Christ, viendra, lui que Jean a prédit être l’Agneau qui doit venir.}
\psaume{9i}{5}
\gscore[n]{A1F1N2A2}{A}{5}{}{}
\gscore{A1F1N2A3}{A}{6}{Ecce venio cito}{Voici que je viens bientôt, et ma récompense est avec moi, dit le Seigneur ; c’est de donner à chacun selon ses œuvres.}
\psaume{9ii}{6}
\gscore[n]{A1F1N2A3}{A}{6}{}{}

\begin{paracol}{2}
\versiculus{Emítte Agnum, Dómine, Dominatórem terræ.}{De Petra desérti ad montem fíliæ Sion.}
\switchcolumn
\versiculus{Envoie, Seigneur, l’Agneau dominateur de la terre.}{De la pierre du désert à la montagne de la fille de Sion.}
\end{paracol}

\nocturn{3}
\gscore{A1F1N3A1}{A}{7}{Gabriel Angelus locutus est}{L’Ange Gabriel parla à Marie, disant : Je vous salue, pleine de grâce, le Seigneur est avec vous, vous êtes bénie entre les femmes.}
\psaume{9iii}{7}
\gscore[n]{A1F1N3A1}{A}{7}{}{}
\gscore{A1F1N3A2}{A}{8}{Maria dixit}{Marie dit: Quelle pensez-vous que soit cette salutation? Parce que mon âme a été troublée, et que je dois enfanter un Roi qui ne prendra pas ma virginité.}
\psaume{9iv}{8}
\gscore[n]{A1F1N3A2}{A}{8}{}{}
\gscore{A1F1N3A3}{A}{9}{In adventu summi Regis}{En l’avènement du souverain Roi, que les cœurs des hommes soient purifiés afin que nous marchions à sa rencontre d’une manière digne: car voici qu’il vient et il ne tardera pas.}
\psaume{10}{4e}
\gscore[n]{A1F1N3A3}{A}{9}{}{}

\begin{paracol}{2}
\versiculus{Egrediétur Dóminus de locl sancto ejus.}{Véniet, ut salvet pópulum suum.}
\switchcolumn
\versiculus{Le Seigneur sortira de son lieu saint.}{Il viendra pour sauver son peuple.}
\end{paracol}

\feast{HF1}{En-dehors de l'Avent et du Temps Pascal}
	{Psautier}{En-dehors de l'Avent et du T.P.}{2}{}{}{}{}{}{}
\addcontentsline{toc}{section}{En-dehors de l'Avent et du Temps Pascal}

\rubric{Dimanches pendant l'année, l'hiver}
\gscore{F1Iw}{I}{}{Adoremus Dominum}{Adorons le Seigneur, car c'est lui qui nous a faits.}

\needspace{6\baselineskip}
\rubric{Dimanches du temps de la Septuagésime.}
\gscore{7GF1I}{I}{}{Praeoccupemus faciem}{Approchons nous devant la face du Seigneur: et acclamons le joyeusement dans les psaumes.}

\rubric{Dimanches de Carême.}
\gscore{Q1I}{I}{}{Non sit vobis vanum}{Ne pensez point que ce soit chose vaine de vous lever matin, avant le jour: car le Seigneur a promis la couronne à ceux qui veillent.}

\rubric{Dimanches au temps de la Passion.}
\gscore{Q5I}{I}{}{Hodie si vocem}{Aujourd’hui, si vous entendez la voix du Seigneur, n’endurcissez pas vos cœurs.}

\rubric{Dimanches pendant l'année, l'été.}
\gscore{F1Is}{I}{}{Dominum qui fecit nos}{Le Seigneur qui nous a faits, venez, adorons-le.}

\rubric{Dimanches pendant l'année, l'hiver, et au temps de la Septuagésime.}
\gscore{F1Hwmed}{H}{}{Primo dierum omnium}{%
\rubric{1.} C'est le premier des jours, jour où la Trinité,
dans sa béatitude a créé l'univers,
où le Créateur, en ressuscitant,
a terrassé la mort et délivré le monde.\\\\
\rubric{2.} Bannissons loin de nous la tiédeur,
levons-nous tous, levons-nous sans retard,
du sein de la nuit, invoquons le Seigneur,
c'est le Prophète-roi qui nous parle et nous presse.\\\\
\rubric{3.} Dieu entendra notre prière,
il nous tendra une main secourable,
purifiera notre âme des souillures
et nous rendra nos droits au Paradis.\\\\
\rubric{4.} Nous qui venons,
en cette très sainte partie du jour,
chanter nos cantiques, durant les heures du repos,
nous aurons part aux récompenses éternelles.\\\\
\rubric{5.} Ô Jésus, splendeur du Père,
nous t'en supplions instamment,
éteins en nous la flamme des passions,
et garde-nous de toute action coupable.\\\\
\rubric{6.} Garde nos corps et nos âmes
du souffle impur de la concupiscence,
c'est à cause de ses feux,
que les feux de l'enfer brûlent avec plus d'ardeur.\\\\
\rubric{7.} Ô Rédempteur du monde, nous t'en supplions
purifie-nous, lave-nous de nos crimes,
et dans ta miséricorde,
accorde-nous les biens de l'éternelle vie.\\\\
\rubric{8.} Là d'où nous fûmes exilés par notre péché,
accueille-nous dans le futur,
en attendant cet heureux temps,
chantons nos mélodies de gloire.\\\\
\rubric{9.} Exauce-nous, Père très miséricordieux,
Fils unique égal au Père,
et toi, Esprit consolateur,
qui règnes dans tous les siècles.
Amen.}

\rubric{Dimanches de Carême.}
\gscore{Q1Hmed}{H}{}{Ex more docti mystico}{%
\rubric{1.} Fidèles à la tradition mystérieuse,
gardons avec soin ce jeûne célèbre
qui parcourt le cercle
de dix jours, quatre fois répétés.\\\\
\rubric{2.} La Loi et les Prophètes
l'inaugurèrent autrefois;
auteur et roi de toutes les choses créées,
le Christ daigna lui-même le consacrer.\\\\
\rubric{3.} Soyons donc d'une plus grande réserve
dans l'usage de la parole, du manger et du boire,
du sommeil et des délassements,
veillons plus strictement sur la garde de nous-mêmes.\\\\
\rubric{4.} Evitons ces périls
où succombe l'âme inattentive;
gardons de laisser la moindre entrée
à notre tyran perfide.\\\\
\rubric{5.} Fléchissons la colère vengeresse;
pleurons aux pieds de notre juge;
poussons des cris suppliants, et,
prosternés devant notre juge, disons-lui:\\\\
\rubric{6.} Ô Dieu par nos péchés,
nous avons offensé ta clémence;
daigne étendre sur nous
ton pardon.\\\\
\rubric{7.} Souviens-toi que, malgré notre fragilité,
nous sommes l'œuvre de tes mains;
ne cède pas à un autre
l'honneur de ton nom.\\\\
\rubric{8.} Pardonne-nous le mal que nous avons fait;
donne-nous avec abondance la grâce que nous implorons,
afin que nous puissions te plaire
ici-bas et dans l'éternité.\\\\
\rubric{9.} Trinité bienheureuse,
Unité parfaite,
rends profitable à tes fidèles
le bienfait du jeûne. Amen.}

\rubric{Dimanches au temps de la Passion.}
\gscore{Q5Hmed}{H}{}{Pange lingua... Proelium}{%
\rubric{1.} Chante, ma langue
la lutte et le glorieux combat;
célèbre le noble triomphe
dont la croix est le trophée,
et la victoire que le Rédempteur du monde
remporta en s'immolant.\\\\
\rubric{2.} Dieu compatit au malheur
du premier homme sorti de ses mains.
Dès que, mordant à la pomme funeste
Adam se précipita dans la mort,
Dieu lui-même désigna l'arbre nouveau
pour réparer les malheurs causés par le premier.\\\\
\rubric{3.} Tel fut le plan divin
dressé pour notre salut,
afin que la sagesse y déjouât
la ruse de notre cauteleux ennemi,
et que le remède nous arrivât par le moyen même
qui avait servi pour nous faire la blessure.\\\\
\rubric{4.} Lors donc que le temps marqué
par le décret divin fut arrivé,
celui par qui le monde a été créé
fut envoyé du trône de son Père,
et ayant pris chair au sein d'une Vierge,
il parut en ce monde.\\\\
\rubric{5.} Petit enfant, il vagit
couché dans une pauvre crèche,
la Vierge, sa Mère enveloppe de langes
ses membres délicats,
et des bandelettes étroites serrent
les mains et les pieds d'un Dieu.\\\\
\rubric{6.} Que toujours en sa béatitude
à la Trinité soit la gloire,
également au Père et au Fils;
pareil honneur au Paraclet:
que du Dieu trine et un, le nom
soit loué dans tout l’Univers.
Amen.}

\rubric{Dimanches pendant l'année, l'été.}
\gscore{F1Hsmed}{H}{}{Nocte surgentes}{%
\rubric{1.} Levons-nous et veillons aux heures de la nuit,
méditons en chantant les psaumes,
unissons doucement nos voix pour offrir au Seigneur
le tribut de nos hymnes.\\\\
\rubric{2.} Chantons, chantons avec les anges,
les douceurs du divin Roi;
chantons pour mériter une part dans le ciel
au festin éternel de la vie.\\\\
\rubric{3.} Écoute-nous, Dieu, Père tout puissant,
écoute-nous, Fils unique du Père,
Esprit Saint, écoute-nous, toi dont la gloire
retentit en l'univers entier.
Amen.}

\nocturn{1}
\gscore{F1N1A1}{A}{1}{Beatus vir... meditatur}{Bienheureux l'homme qui médite la loi du Seigneur.}
\psaume{1}{8}
\gscore[n]{F1N1A1}{A}{1}{}{}
\gscore{F1N1A2}{A}{2}{Servite Domino}{Servez le Seigneur dans la crainte, et exultez devant lui en tremblant.}
\psaume{2}{7}
\gscore[n]{F1N1A2}{A}{2}{}{}
\gscore{F1N1A3}{A}{3}{Exsurge Domine salvum}{Lève-toi, Seigneur, sauve-moi, mon Dieu.}
\psaume{3}{6}
\gscore[n]{F1N1A3}{A}{3}{}{}

\rubric{Pendant l'année.}
\begin{paracol}{2}
\versiculus{Memor fui nocte nóminis tui, Dómine.}{Et custodívi legem tuam.}
\switchcolumn
\versiculus{Je me suis rappelé, la nuit, ton nom, ô Seigneur.}{Et j’ai gardé ta loi.}
\end{paracol}

\newpage
\rubric{En Carême.}
\begin{paracol}{2}
\versiculus{Ipse liberávit me de láqueo venántium.}{Et a verbo áspero.}
\switchcolumn
\versiculus{C’est lui qui m’a délivré du lacet des chasseurs.}{Et de la parole maléfique.}
\end{paracol}

\rubric{Au Temps de la Passion.}
\begin{paracol}{2}
\versiculus{Erue a frámea, Deus, ánimam meam.}{Et de manu canis únicam meam.}
\switchcolumn
\versiculus{Délivre mon âme du glaive, ô Dieu.}{Et de l’atteinte du chien, mon unique.}
\end{paracol}

\nocturn{2}
\gscore{F1N2A1}{A}{4}{Quam admirabile}{Qu'il est admirable ton nom, Seigneur, par toute la terre!}
\psaume{8}{1}
\gscore[n]{F1N2A1}{A}{4}{}{}
\gscore{F1N2A2}{A}{5}{Sedisti super thronum}{Tu sièges sur le trône, toi qui juges avec justice.}
\psaume{9i}{8}
\gscore[n]{F1N2A2}{A}{5}{}{}
\gscore{F1N2A3}{A}{6}{Exsurge Domine non praevaleat}{Lève-toi, Seigneur, que l'homme ne triomphe pas.}
\psaume{9ii}{1}
\gscore[n]{F1N2A3}{A}{6}{}{}

\rubric{Pendant l'année.}
\begin{paracol}{2}
\versiculus{Média nocte surgébam ad confiténdum tibi.}{Super judícia justificatiónis tuæ.}
\switchcolumn
\versiculus{Au milieu de la nuit, je me lève pour te louer.}{Au sujet des jugements de ta justice.}
\end{paracol}

\rubric{En Carême.}
\begin{paracol}{2}
\versiculus{Scápulis suis obumbrábit tibi.}{Et sub pennis ejus sperábis.}
\switchcolumn
\versiculus{Sous ses épaules, il t’abritera.}{Et sous ses ailes, tu auras confiance.}
\end{paracol}

\newpage
\rubric{Au Temps de la Passion.}
\begin{paracol}{2}
\versiculus{De ore leónis líbera me, Dómine.}{Et a cónibus unicónium humilitátem meam.}
\switchcolumn
\versiculus{De la gueule du lion, délivre-moi. Seigneur.}{Et des cornes des buffles, ma faiblesse.}
\end{paracol}

\nocturn{3}
\gscore{F1N3A1}{A}{7}{Ut quid Domine}{Pourquoi, Seigneur, te tenir à l'écart?}
\psaume{9iii}{2}
\gscore[n]{F1N3A1}{A}{7}{}{}
\gscore{F1N3A2}{A}{8}{Exsurge Domine Deus exaltetur}{Lève-toi, Seigneur Dieu, que soit exaltée ta main.}
\psaume{9iv}{5}
\gscore[n]{F1N3A2}{A}{8}{}{}
\gscore{F1N3A3}{A}{9}{Justus Dominus... dilexit}{Juste est le Seigneur et il aime la justice.}
\psaume{10}{1}
\gscore[n]{F1N3A3}{A}{9}{}{}

\rubric{Pendant l'année.}
\begin{paracol}{2}
\versiculus{Prævenérunt óculi mei ad te dilúculo.}{Ut meditárer elóquia tua, Dómine.}
\switchcolumn
\versiculus{Mes yeux se sont hâtés vers toi dès l'aurore.}{Pour que je médite tes paroles, Seigneur.}
\end{paracol}

\rubric{En Carême.}
\begin{paracol}{2}
\versiculus{Scuto circúmdabit te véritas ejus.}{Non timébis a timóre noctúrno.}
\switchcolumn
\versiculus{D’un bouclier, elle te couvrira, sa vérité.}{Et tu ne craindras pas la terreur de la nuit.}
\end{paracol}

\rubric{Au Temps de la Passion.}
\begin{paracol}{2}
\versiculus{Ne perdas cum impiis, Deus, ánimam meam.}{Et cum viris sánguinum vitam meam.}
\switchcolumn
\versiculus{Ne laisse pas mon âme se perdre avec les impies, ô Dieu.}{Ni ma vie avec les hommes de sang.}
\end{paracol}

\feast{PF1}{Temps Pascal}
	{Psautier}{Temps Pascal}{2}{}{}{}{}{}{}
\addcontentsline{toc}{section}{Temps Pascal}

\gscore{P0I}{I}{}{Surrexit Dominus vere}{Le Seigneur est vraiment ressuscité, alléluia.}

\gscore{P1Hmed}{H}{}{Rex sempiterne Domine}{
\rubric{1.} Éternel Roi des habitants des Cieux,
Créateur de l'univers,
Fils de Dieu qui avant tous les siècles
fus toujours égal au Père.\\\\
\rubric{2.} Lorsque le monde naquit à ta parole,
artisan de l'homme, tu donnas à Adam
tos propres traits ; et ta puissance réunit en lui
un noble esprit à un corps sorti de la poussière.\\\\
\rubric{3.} L'envie et l'artifice du démon
entraînèrent bientôt la race humaine dans une dégradation honteuse;
revêtu de la chair, tu es venu
rétablir l'œuvre perdue dont tu avais été l'ouvrier.\\\\
\rubric{4.} Né d'abord de la Vierge,
en ces jours tu renais du sépulcre;
et nous qui étions déjà ensevelis
tu nous commandes de nous lever d'entre les morts.\\\\
\rubric{5.} Pasteur éternel,
tu laves ton troupeau dans l'eau baptismale;
cette eau est la fontaine où se purifient les âmes;
elle est le tombeau où disparaît le péché.\\\\
\rubric{6.} Attaché comme Rédempteur à la croix
qui nous était due,
tu as prodigué ton sang,
la rançon de notre salut.\\\\
\rubric{7.} Pour être à jamais, ô Jésus,
la joie pascale de nos âmes,
daigne sauver de la cruelle mort du péché
ceux que tu as fait renaître à la vie.\\\\
\rubric{8.} À Dieu le Père soit la gloire,
ainsi qu'au Fils ressucité des morts,
et au Consolateur,
durant les siècles éternels.
Amen.}

\nocturn{1}
\gscore{P1F1N1A}{A}{1}{Alleluia lapis revolutus est}{Alléluia, la pierre a été roulée, allélia: dégageant la porte du sépulcre, alléluia, alléluia.}
\psaume{1}{5}
\psaume{2}{5}
\psaume{3}{5}
\gscore[n]{P1F1N1A}{A}{1}{}{}

\begin{paracol}{2}
\versiculus{Surréxit Döminus de sepúlcro, allelúja.}{Qui pro nobis pepéndit in ligno, allelúja.}
\switchcolumn
\versiculus{Le Seigneur s’est levé du sépulcre, alléluia.}{Lui qui pour nous a été suspendu au gibet, alléluia.}
\end{paracol}

\nocturn{2}
\gscore{P1F1N2A}{A}{2}{Alleluia quem quaeris mulier}{Alléluia, qui cherches-tu, ô femme, alléluia; celui qui vit avec les morts, alléluia, alléluia.}
\psaume{8}{5}
\psaume{9i}{5}
\psaume{9ii}{5}
\gscore[n]{P1F1N2A}{A}{2}{}{}

\begin{paracol}{2}
\versiculus{Surréxit Dóminus vere, allelúja.}{Et appáruit Simóni, allelúja.}
\switchcolumn
\versiculus{Le Seigneur est vraiment ressuscité, alléluia.}{Et il est apparu à Simon, alléluia.}
\end{paracol}

\nocturn{3}
\gscore{P1F1N3A}{A}{3}{Alleluia noli flere Maria}{Alléluia, ne pleure plus, Marie, alléluia; le Seigneur est ressuscité, alléluia, alléluia.}
\psaume{9iii}{5}
\psaume{9iv}{5}
\psaume{10}{5}
\gscore[n]{P1F1N3A}{A}{3}{}{}

\begin{paracol}{2}
\versiculus{Gavísi sunt discípuli, allelúja.}{Viso Dómino, allelúja.}
\switchcolumn
\versiculus{Les disciples se réjouirent, alléluia.}{À la vue du Seigneur, alléluia.}
\end{paracol}

\feast{TC}{Tons communs}
	{Tons communs}{Tons communs}{1}{}{}{}{}{}{}
\addcontentsline{toc}{chapter}{Tons communs}

\feast{TCTD}{Te Deum}
	{Tons communs}{Te Deum}{2}{}{}{}{}{}{}
	
\newcommand{\tedeumtranslation}{Nous te louons ô Dieu : nous te reconnaissons pour le Seigneur.
Ô Père éternel, toute la terre te révère.
Tous les Anges les Cieux, et toutes les Puissances,
les Chérubins et les Séraphins te proclament sans cesse :\\\\\\\\
Saint, Saint, Saint le Seigneur, le Dieu des armées.
Les Cieux et la terre sont remplis de la majesté de ta gloire.\\\\\\\\
Le chœur glorieux des Apôtres,
le phalange vénérable des Prophètes,
l'armée des Martyrs éclatante de blancheur célèbre tes louanges;\\\\\\\\
La sainte Église confesse ton nom par toute la terre,
ô Père d'infinie majesté!
Et elle vénère ton Fils véritable et unique,
ainsi que le Saint-Esprit consolateur.\\\\\\\\
Tu es le Roi de gloire ô Christ!
Tu es du Père le Fils éternel.\\\\\\\\
Pour délivrer l'homme, tu n'as pas eu horreur du sein d'une Vierge.
Tu as brisé l'aiguillon de la mort et ouvert aux fidèles le royaume des cieux.
Tu es assis à la droite de Dieu dans la gloire du Père.
Nous croyons que tu es le juge qui doit venir.\\\\\\\\
\rubric{À \normaltext{Te ergo}, on s'agenouille.}\\
Nous te supplions donc de secourir tes serviteurs que tu as rachetés par ton Sang précieux.
Fais qu'ils soient au nombre des saints, dans la gloire éternelle.\\\\\\\\
Sauve ton peuple, Seigneur  et bénis ton héritage.
Conduis tes serviteurs et élèves-les jusque dans l'éternité.
Chaque jour nous te bénissons.
Et nous louons ton nom dans les siècles; et dans les siècles des siècles.\\\\\\\\
Daigne Seigneur, en ce jour nous préserver de tout péché.
Aie pitié de nous Seigneur, aie pitié de nous.\\\\\\\\
Que ta miséricorde, Seigneur se répande sur nous, comme notre espérance est en toi.
J'ai éspéré en toi Seigneur ; que je ne sois pas confondu à jamais.}

\gscore{ORTDa}{H}{}{Te Deum laudamus!Ton simple}{%
\tedeumtranslation
}
\gscore{ORTDb}{H}{}{Te Deum laudamus!Ton solennel}{
\tedeumtranslation
}

\feast{TCBD}{Tons du \emph{Benedicamus Domino}}
	{Tons communs}{Tons communs}{2}{}{}{}{}{}{}
\rubric{Aux solennités}
\gscore{ORBDa}{T}{}{Benedicamus Domino!Solennités}{}

\rubric{Aux fêtes de la Vierge}
\gscore{ORBDd}{T}{}{Benedicamus Domino!Fêtes de la Vierge}{}

\rubric{Le dimanche pendant l'année}
\gscore{ORBDe}{T}{}{Benedicamus Domino!Dimanche pendant l'année}{}

\rubric{Pendant l'Octave de Pâques}
\gscore{ORBDj}{T}{}{Benedicamus Domino!Octave de Pâques}{}

\rubric{Au Temps Pascal}
\gscore{ORBDk}{T}{}{Benedicamus Domino!Temps Pascal}{}

\rubric{Les dimanches de l'Avent et du Carême}
\gscore{ORBDm}{T}{}{Benedicamus Domino!Dimanches de l'Avent et du Carême}{}

%\feast{TCAA}{Tons de l'\emph{Alleluia} à la fin des antiennes}
%	{Tons communs}{Tons communs}{2}{}{}{}{}{}{}
%\gscore[n]{ORAL1}{T}{}{Tons de l'\emph{Alleluia} à la fin des antiennes}{}

\newpage

\feast{TCPA}{Tons des \emph{pneumata} à la fin des antiennes}
	{Tons communs}{Tons communs}{2}{}{}{}{}{}{}
\rubric{Les \emph{pneumata} peuvent être ajoutés à la fin de certaines antiennes, aux grandes fêtes, là où c'est la coutume.}
\gscore[n]{ORAL3}{T}{}{Tons des \emph{pneumata} à la fin des antiennes}{}

\end{document}