% !TEX TS-program = lualatex
% !TEX encoding = UTF-8

\documentclass[11pt, twoside, french]{book}

\input{common_headers}

%% outputs a score with label, indexing, and annotations. no initials if [n] is passed
\makeatletter
\renewcommand{\gscore}[5][y]{
  %% #1 (passed as option) : y = initial, n = no initial
  %% #2 : name of the score file, should be a code, e.g. Q4F4A3 or 1225N1R1
  %% #3 : office-part among the values: T, H, A, P, R, I (toni communes, hy., ant., psalmus, resp., invit.)
  %% #4 : if applicable, a number between 1 and 9 (rank of the ant./resp.) - else: empty
  %% #5 : the indexed name of the piece
  
  %% this prevents page breaks between the phantom section and its label, and the actual score.
  \needspace{4\baselineskip} 
  \protected@edef\@currentlabelname{#5}
  \phantomsection
  \label{#2}
  %% we add the office part, and number of that ant. or resp. in the current office, if applicable
  %% todo : the negative hspace is here because somehow the initial and annotation (first line only) are misaligned by 1mm _with this initial font size_.
  %% this should probably be fixed in a more elegant way.
  \greannotation[c]{\hspace{-1.4mm}\hspace{\defaultannotationshift}\officepartannotation{#3}#4}
  %% if #5 (indexed name) is blank, nothing is indexed.
  %% this is for pieces that are repetitions of another piece (antiphons after psalms)
  \ifblank{#5}{}{\index[#3]{#5}}
  %% if optional arg #1 has been passed as 'n', set no initial
  \ifx n#1\gresetinitiallines{0}\fi
  %% the use of a directory called "gabc" is linked
  %% to the management of gabc files by the website: do not change 
  %% without also changing the website static files structure
  \gregorioscore{\subfix{nocturnale-romanum/gabc/#2}}
  %% if optional arg #1 has been passed as 'n', unset no initial
  \ifx n#1\gresetinitiallines{1}\fi
  \vspace{1mm}
}
\makeatother

\newcommand{\psaume}[2]{
	\vspace{0.5\baselineskip}
	\needspace{5\baselineskip}
	\phantomsection
	\label{Psalm#1_#2}
	\index[P]{#1 (mode #2)}
	\smalltitle{Psaume #1}
	\begin{paracols}{2}
	\begin{itemize}[
		label=\null, 
		leftmargin=0pt, 
		itemindent=0pt, 
		labelsep=0pt, 
		labelwidth=0pt, 
		rightmargin=0pt, 
		parsep=0pt, 
		itemsep=0pt]
	\input{nocturnale-romanum/psalmi/#1_#2.tex}
	\end{itemize}
	\switchcolumn
	\begin{itemize}[
		label=\null, 
		leftmargin=0pt, 
		itemindent=0pt, 
		labelsep=0pt, 
		labelwidth=0pt, 
		rightmargin=0pt, 
		parsep=0pt, 
		itemsep=0pt]
	\input{psaumes_fr/#1.tex}
	\end{itemize}
	\end{paracols}
}

\newcommand{\translation}[1]{
    \emph{#1}
}

\begin{document}

\customsubfile{nr01_prolegomena}
\customsubfile{nr10_tempus_adventus}
\customsubfile{nr11_tempus_nativitatis}
\customsubfile{nr12_tempus_post_epiphaniam}
\customsubfile{nr13_tempus_quadragesimae}
\customsubfile{nr14_tempus_passionis}
\customsubfile{nr15_tempus_paschale}
\customsubfile{nr16_tempus_post_pentecosten}
\customsubfile{nr17_tempus_augusti_septembri}
\customsubfile{nr18_tempus_octobri_novembri}
\customsubfile{nr20_psalterium_ordinarium}
\customsubfile{nr21_psalterium_hebdomada}
\customsubfile{nr30_commune_apostolorum}
\customsubfile{nr31_commune_martyrum}
\customsubfile{nr32_commune_confessoris}
\customsubfile{nr33_commune_mulierum}
\customsubfile{nr34_commune_dedicationis}
\customsubfile{nr35_commune_bmv}
\customsubfile{nr36_officium_defunctorum}
\customsubfile{nr40_sanctorale_januarii}
\customsubfile{nr41_sanctorale_februarii}
\customsubfile{nr42_sanctorale_martii}
\customsubfile{nr43_sanctorale_aprilis}
\customsubfile{nr44_sanctorale_maii}
\customsubfile{nr45_sanctorale_junii}
\customsubfile{nr46_sanctorale_julii}
\customsubfile{nr47_sanctorale_augusti}
\customsubfile{nr48_sanctorale_septembris}
\customsubfile{nr49_sanctorale_octobris}
\customsubfile{nr50_sanctorale_novembris}
\customsubfile{nr51_sanctorale_decembris}

\printindex[I]
\printindex[H]
\printindex[A]
\printindex[R]
\printindex[P]
\printindex[T]
\printindex[F]

\end{document}

